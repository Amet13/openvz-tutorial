\section{Создание и настройка нового сервера VPS}
\subsection{Проверка сети и дискового пространства}

После загрузки в \texttt{vzkernel}, первым делом необходимо проверить настройки сети:
\begin{lstlisting}
# ifconfig | grep "eth\|lo\|venet" -A 1
eth0      Link encap:Ethernet  HWaddr 08:00:27:14:41:DE  
          inet addr:192.168.0.100  Bcast:192.168.0.255  Mask:255.255.255.0
--
lo        Link encap:Local Loopback  
          inet addr:127.0.0.1  Mask:255.0.0.0
--
venet0    Link encap:UNSPEC  HWaddr 00-00-00-00-00-00-00-00-00-00-00-00-00-00-00-00  
          inet6 addr: fe80::1/128 Scope:Link
\end{lstlisting}

Должно быть доступно три сетевых интерфейса:
\begin{itemize}
    \item \texttt{eth0 (192.168.0.100)} "--- интерфейс реальной сетевой карты;
    \item \texttt{lo (127.0.0.1)} "--- виртуальный интерфейс локальная <<петля>>;
    \item \texttt{venet0} "--- виртуальный сетевой интерфейс для OpenVZ.
\end{itemize}

Проверим свободное место на диске:
\begin{lstlisting}
# df -h | grep vz
/dev/sda2        32G  1.1G   30G   4% /vz
\end{lstlisting}

По этому примеру видно, что всего доступно 32G дискового пространства на разделе \texttt{/vz}, из них занято 1.1G, скачанными ранее шаблонами ОС.
С учетом того, что на один контейнер будем тратить не более 1G дискового пространства, то 30G должно хватить.

\subsection{Идентификаторы контейнеров}
При создании, каждый контейнер имеет идентификатор (CTID)\footnote{CTID "--- ConTainer IDentificator}:
\begin{itemize}
    \item CTID с номером 0 "--- это хост-компьютер;
    \item CTID от 1 до 100 резервируются OpenVZ для внутренних нужд.
\end{itemize}

CTID должен быть уникальным для каждого контейнера.
Хорошей практикой является создание контейнера с CTID от 101 до 999.

Существует также схема присвоения идентификаторов по IP адресам.
Например, для адреса \texttt{10.0.2.1}, CTID = 102, для \texttt{192.168.123.33}, CTID = 33123 и~т.~д. \cite{virtuozzolinux}

\subsection{Просмотр списка контейнеров и шаблоны}
Посмотрим список созданных контейнеров.
Так как, пока VPS не созданы, то можно увидеть такое:
\begin{lstlisting}
# vzlist -a
      CTID      NPROC STATUS    IP_ADDR         HOSTNAME 
\end{lstlisting}

Для создания контейнера нужно использовать один из ранее скачанных шаблонов.
Проверим, какие шаблоны доступны на хост-компьютере:
\begin{lstlisting}
# vztmpl-dl --list-local
centos-6-x86_64-minimal
centos-7-x86_64-minimal
debian-7.0-x86_64-minimal
suse-13.1-x86_64-minimal
ubuntu-14.04-x86_64-minimal
\end{lstlisting}

\subsection{Конфигурационные файлы}
Каждый контейнер имеет свой конфигурационный файл (далее <<конфиг>>), который хранится в каталоге \texttt{/etc/sysconfig/vz-scripts/}.

Именуются конфиги по CTID контейнера.
Например, для контейнера с CTID = 101, конфиг будет называться \texttt{101.conf}.

При создании контейнера можно использовать типовую конфигурацию для VPS.
Типовые файлы конфигураций находятся в том же каталоге \texttt{/etc/sysconfig/vz-scripts/}:
\begin{lstlisting}
# ls -1 /etc/sysconfig/vz-scripts/ | grep sample
ve-basic.conf-sample
ve-custom.conf-sample
ve-light.conf-sample
ve-vswap-1024m.conf-sample
ve-vswap-1g.conf-sample
ve-vswap-256m.conf-sample
ve-vswap-2g.conf-sample
ve-vswap-4g.conf-sample
ve-vswap-512m.conf-sample
\end{lstlisting}

В этих конфигурационных файлах описаны контрольные параметры ресурсов, выделенное дисковое пространство, оперативная память и~т.~д.

Например, при использовании конфига \texttt{ve-vswap-1g}, создается VPS с дисковым пространством 2G, оперативной памятью 1G и swap 2G.
Это удобно, так как существует возможность создавать свои конфигурационные файлы для различных вараиций VPS.

Cоздадим свой конфигурационный файл, на базе уже существующего (\texttt{vswap-256m}).
Исправим в нем только значения \texttt{DISKSPACE}, \texttt{PHYSPAGES} и \texttt{SWAPPAGES}:
\begin{lstlisting}
# cp /etc/vz/conf/ve-vswap-256m.conf-sample /etc/vz/conf/ve-custom.conf-sample
# vim /etc/vz/conf/ve-custom.conf-sample
DISKSPACE="1G:1.1G"
PHYSPAGES="0:128M"
SWAPPAGES="0:128M"
\end{lstlisting}

Таким образом, при использовании этого конфигурационного файла, будет создаваться контейнер, которому будет доступен 1G выделенного дискового пространства, 128M оперативной памяти и 128M swap.

В дальнейшем, при создании контейнеров будем использовать конфигурационный файл \texttt{custom}.

\subsection{Создание и настройка контейнера}
Для создания контейнера необходимо ввести команду:
\begin{lstlisting}
# vzctl create 101 --ostemplate debian-7.0-x86_64-minimal --config custom
\end{lstlisting}
где:
\begin{itemize}
    \item \texttt{101} "--- CTID контейнера;
    \item \texttt{debian-7.0-x86\_64-minimal} "--- шаблон ОС;
    \item \texttt{custom} "--- желаемый шаблон конфигурационного файла.
\end{itemize}

После нажатия клавиши \texttt{Enter} начинается процесс создания VPS. 
По времени процедура может занимать несколько десятков секунд.

Проверим правильность создания VPS:
\begin{lstlisting}
# vzlist -a
      CTID      NPROC STATUS    IP_ADDR         HOSTNAME
       101          - stopped   -               -
\end{lstlisting}

Можно увидеть, что создан контейнер с CTID = 101, сейчас он не включен.

Если же при создании контейнера не указывать желаемый шаблон и файл конфигурации, то OpenVZ будет использовать шаблон и конфигурационный файл по умолчанию.
Конфиг, в котором указаны директивы по умолчанию имеет имя: \texttt{/etc/sysconfig/vz}.
По умолчанию, используется шаблон \texttt{centos-6-x86} и конфигурационный файл \texttt{vswap-256m}.

Так как планируется создание небольшого количества VPS, основываясь на одном и том же конфиге, то исправим\footnote{В файле \texttt{/etc/sysconfig/vz} (и многих других) с символа \texttt{\#} начинается комментарий} эти значения на нужные:
\begin{lstlisting}
# vim /etc/sysconfig/vz
#CONFIGFILE="vswap-256m"
CONFIGFILE="custom"
#DEF_OSTEMPLATE="centos-6-x86"
DEF_OSTEMPLATE="debian-7.0-x86_64-minimal"
\end{lstlisting}

Теперь, при создании VPS достаточно указать только CTID контейнера, например:
\begin{lstlisting}
# vzctl create 101
\end{lstlisting}

Будет создан контейнер на базе \texttt{debian-7.0x86\_64-minimal}, значения системных параметров будут взяты с конфига \texttt{custom}.

Контейнер создан, его можно запускать.
Но перед первым запуском необходимо установить его IP адрес, hostname, указать DNS сервер и задать пароль суперпользователя.

%\subsection{Настройка контейнера}
Для настройки VPS используется команда \texttt{vzctl set}.

Для того, чтобы контейнер запускался при старте хост-компьютера (например после перезагрузки), необходимо использовать команду:
\begin{lstlisting}
# vzctl set 101 --onboot yes --save
CT configuration saved to /etc/vz/conf/101.conf
\end{lstlisting}

При использовании ключа \texttt{-{}-save} в \texttt{vzctl set}, сохраняются параметры контейнера в соответствующий конфигурационный файл.

Аналогично можно задать hostname:
\begin{lstlisting}
# vzctl set 101 --hostname stud1 --save
CT configuration saved to /etc/vz/conf/101.conf
\end{lstlisting}

Установка IP адреса:
\begin{lstlisting}
# vzctl set 101 --ipadd 192.168.0.101 --save
CT configuration saved to /etc/vz/conf/101.conf
\end{lstlisting}

Адрес DNS сервера (в большинстве случаев\footnote{Если же нужно явно указать адрес DNS сервера, то вместо \texttt{inherit} можно указать IP адрес, например \texttt{192.168.0.1}} адрес DNS совпадает с адресом хост-компьютера, поэтому можно вместо адреса указать параметр \texttt{inherit}):
\begin{lstlisting}
# vzctl set 101 --nameserver inherit --save
CT configuration saved to /etc/vz/conf/101.conf
\end{lstlisting}

Установка пароля суперпользователя:
\begin{lstlisting}
# vzctl set 101 --userpasswd root:p@ssw0rd
Starting container...
Opening delta /vz/private/101/root.hdd/root.hdd
Adding delta dev=/dev/ploop37965 img=/vz/private/101/root.hdd/root.hdd (rw)
Mounting /dev/ploop37965p1 at /vz/root/101 fstype=ext4 data='balloon_ino=12,' 
Container is mounted
Container start in progress...
Killing container ...
Container was stopped
Unmounting file system at /vz/root/101
Unmounting device /dev/ploop37965
Container is unmounted
\end{lstlisting}

Пароль будет установлен в VPS, в файл \texttt{/etc/shadow} и не будет сохранен в конфигурационный файл контейнера.
Если же пароль будет утерян или забыт, то можно будет просто задать новый.

\subsection{Запуск и вход}
После настроек нового контейнера, его можно запустить:
\begin{lstlisting}
# vzctl start 101
Starting container...
Opening delta /vz/private/101/root.hdd/root.hdd
Adding delta dev=/dev/ploop37965 img=/vz/private/101/root.hdd/root.hdd (rw)
Mounting /dev/ploop37965p1 at /vz/root/101 fstype=ext4 data='balloon_ino=12,' 
Container is mounted
Adding IP address(es): 192.168.0.101
Setting CPU units: 1000
Container start in progress...
\end{lstlisting}

Для того, чтобы выполнить команду внутри контейнера существует команда \texttt{vzctl exec}.
Подробнее об этой команде позже.

Проверяем сетевые интерфейсы внутри гостевой ОС:
\begin{lstlisting}
# vzctl exec 101 ifconfig | grep "lo\|venet" -A 1
lo        Link encap:Local Loopback  
          inet addr:127.0.0.1  Mask:255.0.0.0
--
venet0    Link encap:UNSPEC  HWaddr 00-00-00-00-00-00-00-00-00-00-00-00-00-00-00-00  
          inet addr:127.0.0.2  P-t-P:127.0.0.2  Bcast:0.0.0.0  Mask:255.255.255.255
--
venet0:0  Link encap:UNSPEC  HWaddr 00-00-00-00-00-00-00-00-00-00-00-00-00-00-00-00  
          inet addr:192.168.0.101  P-t-P:192.168.0.101  Bcast:192.168.0.101  Mask:255.255.255.255

\end{lstlisting}

Должны присутствовать сетевые интерфейсы:
\begin{itemize}
    \item \texttt{lo (127.0.0.1)};
    \item \texttt{venet0 (127.0.0.2)};
    \item \texttt{venet0:0 (192.168.0.101)}.
\end{itemize}

Если сеть в порядке, то можно соединиться к контейнеру по SSH с хост-компьютера:
\begin{lstlisting}
# ssh root@192.168.0.101
root@192.168.0.101's password: p@ssw0rd
\end{lstlisting}

Вход в контейнер напрямую с хост-компьютера осуществляется командой \texttt{vzctl enter}:
\begin{lstlisting}
# vzctl enter 101
entered into CT 101
root@stud1:/# 
\end{lstlisting}

Выход из контейнера:
\begin{lstlisting}
root@stud1:/#  exit
logout
exited from CT 101
\end{lstlisting}

\subsection{Статус VPS}
Для того, чтобы узнать статус контейнера, используется команда:
\begin{lstlisting}
# vzctl status 101
CTID 101 exist mounted running
\end{lstlisting}

По выводу команды можно видеть, что контейнер с CTID = 101 существует, смонтирован и запущен.

Команда \texttt{vzlist -a} выводит список всех существующих в системе контейнеров.
Рассмотрим подробно вывод команды \texttt{vzlist -a}:
\begin{lstlisting}
# vzlist -a
      CTID      NPROC STATUS    IP_ADDR         HOSTNAME
       101         15 running   192.168.0.101   stud1
       102          - stopped   192.168.0.102   stud2
\end{lstlisting}
где:
\begin{itemize}
    \item \texttt{CTID} "--- ID контейнера;
    \item \texttt{NPROC} "--- число запущенных процессов в контейнере;
    \item \texttt{STATUS} "--- состояние контейнера (запущен/не запущен);
    \item \texttt{IP\_ADDR} "--- IP адрес;
    \item \texttt{HOSTNAME} "--- имя контейнера.
\end{itemize}

\subsection{Остановка, перезапуск и удаление контейнера}
Для остановки контейнера используется команда:
\begin{lstlisting}
# vzctl stop 101
\end{lstlisting}

Для полной остановки контейнера, системе требуется немного времени (примерно 2 минуты).

Иногда нужно выключить VPS как можно быстрее, например, если контейнер был подвержен взлому.
Для того чтобы срочно выключить VPS, нужно использовать ключ \texttt{-{}-fast}:
\begin{lstlisting}
# vzctl stop 101 --fast
Killing container ...
Container was stopped
Unmounting file system at /vz/root/101
Unmounting device /dev/ploop37965
Container is unmounted
\end{lstlisting}

Для перезапуска контейнера можно использовать команду:
\begin{lstlisting}
# vzctl restart 101
Restarting container
Stopping container ...
Container was stopped
Unmounting file system at /vz/root/101
Unmounting device /dev/ploop37965
Container is unmounted
Starting container...
Opening delta /vz/private/101/root.hdd/root.hdd
Adding delta dev=/dev/ploop37965 img=/vz/private/101/root.hdd/root.hdd (rw)
Mounting /dev/ploop37965p1 at /vz/root/101 fstype=ext4 data='balloon_ino=12,' 
Container is mounted
Adding IP address(es): 192.168.0.101
Setting CPU units: 1000
Container start in progress...
\end{lstlisting}

А для того чтобы удалить контейнер, его нужно сначала остановить:
\begin{lstlisting}
# vzctl stop 101
Stopping container ...
Container was stopped
Unmounting file system at /vz/root/101
Unmounting device /dev/ploop37965
Container is unmounted
\end{lstlisting}

Для удаления используется команда:
\begin{lstlisting}
# vzctl destroy 101
CTID 101 deleted unmounted down
\end{lstlisting}

Команда выполняет удаление частной области сервера и переименовывает файл конфигурации, дописывая к нему \texttt{.destroyed}\footnote{Например, после удаления контейнера с CTID = 101, конфиг стал называться \texttt{/etc/vz/conf/101.conf.destroyed}}.

\subsection{Запуск команд}
Как уже было сказано выше, для запуска команд в контейнере используется команда:
\begin{lstlisting}
# vzctl exec 101 command
\end{lstlisting}

Например, для того, чтобы соединиться к VPS по SSH\footnote{SSH позволяет безопасно передавать в незащищенной среде практически любой сетевой протокол. Таким образом, можно не только удаленно работать на компьютере через командную оболочку, но и передавать по шифрованному каналу звуковой поток или видео. Также SSH может использовать сжатие передаваемых данных для последующего их шифрования}, нужно сначала включить SSH:
\begin{lstlisting}
# vzctl exec 101 service ssh start
Starting OpenBSD Secure Shell server: sshd.
\end{lstlisting}

Теперь можно соединиться к контейнеру по SSH:
\begin{lstlisting}
# ssh root@192.168.0.101
root@192.168.0.101's password: p@ssw0rd
root@stud1:/# 
\end{lstlisting}

Иногда бывает нужно выполнить команду на нескольких VPS.
Для этого можно использовать команду:
\begin{lstlisting}
# for i in `vzlist -o veid -H`; do \
> echo "VPS $i"; vzctl exec $i command; done
\end{lstlisting}

Например, можно узнать, сколько времени работают все запущенные контейнеры:
\begin{lstlisting}
# for i in `vzlist -o veid -H`; do \
> echo "VPS $i"; vzctl exec $i uptime; done
VPS 101
 05:45:01 up 2 min, 0 users, load average: 0.01, 0.02, 0.03
VPS 102
 05:46:01 up 1 min, 0 users, load average: 0.04, 0.05, 0.06
\end{lstlisting}

Или узнать системную информацию о дистрибутивах в контейнерах:
\begin{lstlisting}
# for i in `vzlist -o veid -H`; do \
> echo "VPS $i"; vzctl exec $i uname -rv; done
VPS 101
2.6.32-042stab093.5 #1 SMP Wed Jan 01 12:12:12 MSK 2014
VPS 102
2.6.32-042stab093.5 #1 SMP Wed Jan 01 12:12:12 MSK 2014
\end{lstlisting}

\clearpage
