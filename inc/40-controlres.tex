\section{Управление ресурсами}

Системный оператор контролирует, доступные частному серверу ресурсы, с помощью набора параметров управления ресурсами. 
Все эти параметры можно редактировать в файлах шаблонов, в каталоге \texttt{/etc/sysconfig/vz-scripts/}.
Их можно установить вручную, редактируя соответствующие конфиги или используя утилиты OpenVZ.

Параметры контроля ресурсов условно разделяют на три группы:
\begin{itemize}
    \item Дисковые (управление квотами диска, фрагментацией);
    \item Процессорные (распределение процессорного времени);
    \item Системные (сеть, память).
\end{itemize}

\subsection{Дисковые параметры}
Администратор OpenVZ сервера может установить дисковые квоты, в терминах дискового пространства и количества inodes, число которых примерно равно количеству файлов. 
Это первый уровень дисковой квоты.
В дополнение к этому, администратор может использовать обычные утилиты внутри окружения, для настроек стандартных дисковых квот UNIX для пользователей и групп.

Основные параметры:
\begin{itemize}
    \item \texttt{DISKSPACE} "--- общий размер дискового пространства;
    \item \texttt{DISKINODES} "--- общее число дисковых inodes\footnote{inode (индексный дескриптор) "--- структура данных в традиционных для ОС UNIX файловых системах, таких как UFS. В этой структуре хранится метаинформация о стандартных файлах, каталогах или других объектах файловой системы, кроме непосредственно данных и имени};
    \item \texttt{QUOTATIME} "--- время (в секундах) на которое VPS может превысить значение \texttt{soft} предела.
\end{itemize}

Первые два параметра записываются в виде:
\begin{lstlisting}
COMMAND="softlimit:hardlimit"
\end{lstlisting}
где:
\begin{itemize}
    \item \texttt{COMMAND} "--- команда (\texttt{DISKSPACE} или \texttt{DISKINODES});
    \item \texttt{softlimit} "--- значение которое превышать нежелательно, после пересечения этого предела наступает grace период, по истечении которого, дисковое пространство или inodes прекратят свое существование;
    \item \texttt{hardlimit} "--- значение которое превысить нельзя.
\end{itemize}

Например, запись:
\begin{lstlisting}
DISKSPACE="1G:1.1G"
DISKINODES="100000:110000"
QUOTATIME="600" 
\end{lstlisting}
означает, что задается \texttt{softlimit} для дискового пространства равным 1G и \texttt{hardlimit} равный 1.1G, то же самое с inode 100000 и 110000 соответственно.

Если размер занятого дискового пространства или inodes будет выше \texttt{softlimit}, то в течении 600 сек (10 мин), в случае не освобождения дискового пространства или inodes, они прекратят свое существование.

Аналогично, можно установить эти параметры с помощью \texttt{vzctl}:
\begin{lstlisting}
# vzctl set 101 --diskspace 1G:1.1G --save
\end{lstlisting}
\begin{lstlisting}
# vzctl set 101 --diskinodes 10000:110000 --save
\end{lstlisting}
\begin{lstlisting}
# vzctl set 101 --quotatime 600 --save
\end{lstlisting}

\subsection{Параметры процессора}
Планировщик процессора в OpenVZ также двухуровневый. 
На первом уровне планировщик решает, какому контейнеру дать квант процессорного времени, базируясь на значении параметра \texttt{CPUUNITS} для VPS. 
На втором уровне стандартный планировщик GNU/Linux решает, какому процессу в выбранном контейнере дать квант времени, базируясь на стандартных приоритетах процесса.

Основные параметры:
\begin{itemize}
    \item \texttt{CPUS} "--- целое число, определяющее число процессоров (ядер) для конетйнера;
    \item \texttt{CPULIMIT} "--- верхний лимит процессорного времени в процентах;
    \item \texttt{CPUUNITS} "--- гарантируемое минимальное количество времени процессора, которое получит соответствующий VPS.
\end{itemize}

Задать эти параметры можно как в файле конфигурации контейнера, так и вручную:
\begin{lstlisting}
# vzctl set 101 --cpus 2 --cpulimit 4 --cpuunits 1500 --save
\end{lstlisting}

Утилиты контроля ресурсов процессора, гарантируют любому VPS количество времени центрального процессора, которое собственно и получает этот VPS. 
При этом контейнер может потреблять больше времени, чем определено этой величиной, если нет другого конкурирующего с ним за время CPU сервера.

\subsection{Системные параметры}

В терминах OpenVZ лимиты и гарантии ресурсов называются User Beancounters (UBC). 
Всего существует около 20 UBC, контролирующих почти все возможные ресурсы системы. 
Каждый UBC имеет свою опцию в команде \texttt{vzctl}, а также строку в конфигурационном файле \texttt{/proc/user\_beancounters}, с помощью которого можно узнать о текущем количестве выделенных ресурсов и определить их нехватку. 

Файл представляет собой таблицу, каждая строка которой содержит информацию об одном ресурсе, а колонки отражают следующие данные:
\begin{itemize}
    \item \texttt{uid} "--- идентификатор контейнера;
    \item \texttt{resource} "--- имя ресурса;
    \item \texttt{held} "--- текущая утилизация ресурса;
    \item \texttt{maxheld} "--- максимальный уровень утилизации ресурса за все время работы контейнера;
    \item \texttt{barrier} "--- максимальный уровень утилизации ресурсов, который может быть временно превышен;
    \item \texttt{limit} "--- жесткое ограничение утилизации ресурса, которое никогда не может быть превышено;
    \item \texttt{failcnt} "--- счетчик отказов, который увеличивается каждый раз, когда контейнер делает запрос ресурсов сверх своего лимита.
\end{itemize}

Не обязательно разбираться во всех тонкостях системы подсчета ресурсов OpenVZ, чтобы эффективно управлять контейнерами. 
Достаточно время от времени поглядывать на значение колонки \texttt{failcnt} и, если оно оказывается больше нуля, начинать предпринимать меры либо по оптимизации исполняемого в рамках контейнера ПО, либо по увеличению количества выделяемых контейнеру ресурсов \cite{xakep}.

Начиная с версий ядра RHEL 6 \texttt{042stab04x}, появилась поддержка vSwap \cite{ubc}.
Теперь не нужно высчитывать UBC лимиты, достаточно при создании гостевой системы указать всего лишь \texttt{PHYSPAGES} и \texttt{SWAPPAGES}.

Начиная с ядра \texttt{042stab068.8} появилась возможность ограничивать использование контейнерами дискового кэша.

Более подробную информацию про User Beancounters можно получить по адресам: \url{http://kb.sp.parallels.com/ru/112807} и \url{http://openvz.org/UBC}.

\clearpage
