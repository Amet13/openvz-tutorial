%%% Преамбула %%%

%%% Поля и разметка страницы %%%
\usepackage{lscape}		% Для включения альбомных страниц

%%% Кодировки и шрифты %%%
\usepackage{cmap}						% Улучшенный поиск русских слов в полученном pdf-файле
\usepackage[T2A]{fontenc}				% Поддержка русских букв
\usepackage[utf8]{inputenc}				% Кодировка utf8
\usepackage[english, russian]{babel}	% Языки: русский, английский
\usepackage{pscyr}						% Красивые русские шрифты

%%% Математические пакеты %%%
\usepackage{amsthm,amsfonts,amsmath,amssymb,amscd} % Математические дополнения от AMS

%%% Оформление абзацев %%%
\usepackage{indentfirst} % Красная строка

%%% Цвета %%%
\usepackage[usenames]{color}
\usepackage{color}
\usepackage{colortbl}

%%% Таблицы %%%
\usepackage{longtable}					% Длинные таблицы
\usepackage{multirow,makecell,array}	% Улучшенное форматирование таблиц

%%% Общее форматирование
\usepackage[singlelinecheck=off,center]{caption}	% Многострочные подписи
\usepackage{soul}									% Поддержка переносоустойчивых подчеркиваний и зачеркиваний

%%% Библиография %%%
%\usepackage{cite} % Красивые ссылки на литературу

%%% Гиперссылки %%%
\usepackage[plainpages=false,pdfpagelabels=false]{hyperref}
\definecolor{linkcolor}{rgb}{0.9,0,0}
\definecolor{citecolor}{rgb}{0,0.6,0}
\definecolor{urlcolor}{rgb}{0,0,1}
\hypersetup{
    colorlinks, linkcolor={linkcolor},
    citecolor={citecolor}, urlcolor={urlcolor},
    pdfauthor={Amet Umerov},
    pdftitle={OpenVZ Tutorial},
}

%%% Изображения %%%
\usepackage[dvipdfm]{graphicx}		% Подключаем пакет работы с графикой
\graphicspath{{images/}{images/centos/}}	% Пути к изображениям

%%% Выравнивание и переносы %%%
\sloppy					% Избавляемся от переполнений
\clubpenalty=10000		% Запрещаем разрыв страницы после первой строки абзаца
\widowpenalty=10000		% Запрещаем разрыв страницы после последней строки абзаца

%%% Библиография %%%
\makeatletter
\bibliographystyle{utf8gost705u}	% Оформляем библиографию в соответствии с ГОСТ 7.0.5
\renewcommand{\@biblabel}[1]{#1.}	% Заменяем библиографию с квадратных скобок на точку:
\makeatother

%%% Колонтитулы %%%
\usepackage{fancyhdr}
 	\pagestyle{fancy}
 	\fancyhead[L]{\thepage} % Нумерация страниц в левом верхнем углу
	
\let\Sectionmark\sectionmark
\def\sectionmark#1{\def\Sectionname{#1}\Sectionmark{#1}}
\makeatletter
\newcommand*{\currentname}{\@currentlabelname}
\renewcommand{\@oddhead}{\it \vbox{\hbox to \textwidth%
    {\rm{\thepage \hfil \thesection.~\Sectionname}\strut}\hrule}}
\makeatother
    \fancyhf{} % Очистка текущих значений номеров страниц

%%% Библиография %%%
\usepackage[backend=biber,bibencoding=utf8,sorting=none]{biblatex}
\usepackage{csquotes}
\bibliography{inc/bib1.bib}

%%% Подсветка кода %%%
\usepackage{color}
\usepackage{listings}

\definecolor{backcolor}{rgb}{0.95,0.95,0.95}

\lstset{
  columns=fixed,					% Делаем моноширинный шрифт
  backgroundcolor=\color{backcolor},% Цвет фона, нужно подключить пакет color или xcolor  
  basicstyle=\small\ttfamily,		% Размер и начертание  
  breakatwhitespace=false,			% Переносим строки только при наличии пробела  
  breaklines=true,					% Автоматически переносим строки    
  commentstyle=\color{black},	    % Стиль для комментариев   
  extendedchars=true,				% Позволяем использовать не-ASCII символы  
  keywordstyle=\color{black},		% Стиль для ключевых слов 
  language=bash,					% Язык программирования  
  morekeywords={*,...},				% Пользовательские ключевые слова  
  showspaces=false,					% Показывать ли пробелы с помощью специальных отступов  
  showstringspaces=false,			% Показывать ли пробелы в строках  
  showtabs=false,					% Показывать ли знаки табуляции
  stepnumber=1,						% Размер шага между номерами строк
  tabsize=2						    % Размер табуляции
}

%%% Счетчик номера сноски на каждой странице сбрасывается %%%
\usepackage{perpage}
\MakePerPage{footnote}

\usepackage{afterpage}
