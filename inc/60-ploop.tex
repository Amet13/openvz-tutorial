\section{ploop}

Для работы OpenVZ с файлами контейнера, существует два метода:
\begin{itemize}
    \item simfs (каталоги и файлы в файловой системе хост-компьютера);
    \item ploop (отдельный файл для каждого контейнера).
\end{itemize}

simfs уже осталась в прошлом, в последнее время ее вытеснил ploop, который готов для использования в production\footnote{production "--- производственная <<боевая>> среда}.

По умолчанию в OpenVZ используется ploop.
Настраивается это в конфигурационном файле \texttt{/etc/vz/vz.conf}:
\begin{lstlisting}
# cat /etc/vz/vz.conf | grep VE_LAYOUT
VE_LAYOUT=ploop
\end{lstlisting}

Основные преимущества ploop \cite{ploop}:
\begin{itemize}
%    \item Нет ограничения на индексные дескрипторы;
    \item Поддержка корректной и надежной изоляции пользователей друг от друга;
    \item Простое резервное копирование;
    \item Журнал файловой системы больше не является узким местом;
    \item Живая миграция;
    \item Поддержка различных типов хранения данных.
\end{itemize}

В ploop игнорируются параметры \texttt{DISKQUOTA}, \texttt{DISKINODES}, \texttt{QUOTATIME}.
Параметр \texttt{DISKSPACE} не игнорируется.

Сами диски хранятся в каталоге \texttt{/vz/private} и имеют имя \texttt{root.hdd}:
\begin{lstlisting}
# ls -1 /vz/private/*/root.hdd/root.hdd
/vz/private/101/root.hdd/root.hdd
/vz/private/102/root.hdd/root.hdd
/vz/private/103/root.hdd/root.hdd
\end{lstlisting}

\iffalse
Существует скрипт для конвертации simfs в ploop. 
Для его использования, на хост-компьютере должен быть установлен git\footnote{\texttt{yum install git}}.

Работа со скриптом:
\begin{lstlisting}
# git clone https://github.com/FastVPSEestiOu/simfs_to_ploop/
# cd simfs_to_ploop/
\end{lstlisting}

\begin{lstlisting}
# perl fast_convert_to_ploop.pl
Process CT 104
CT inodes 104 used 14831 max: 144179
CT bytes 104 used 249520 max: 1153433
We should stop 104 container (current status: running)
Start ploop conversion for: 104
Creating image: /vz/private/104.ploop/root.hdd/root.hdd size=1153433K
Creating delta /vz/private/104.ploop/root.hdd/root.hdd bs=2048 size=4614144 sectors v2
...
\end{lstlisting}

Скрипт автоматически находит все контейнеры, использующие simfs, останавливает их, конвертирует в ploop и запускает.
Для конвертации используется команда \texttt{vzctl convert}.
\fi

Можно быстро изменять размер ploop-диска, не отключая при этом контейнер.
Просмотрим размеры ploop-дисков (1G):
\begin{lstlisting}
# vzlist -o smart_name,diskspace.h
 SMARTNAME   DSPACE.H
       101    1113684
       102    1113684
       103    1113684
       104    1113684
\end{lstlisting} 

Изменим размер диска контейнера 103 на 2G:
\begin{lstlisting}
# vzctl set 103 --diskspace 2G --save
dumpe2fs 1.41.12 (17-May-2010)
Changing balloon size old_size=1182793728 new_size=217055232
Successfully truncated balloon from 1182793728 to 217055232 bytes
UB limits were set successfully
CT configuration saved to /etc/vz/conf/103.conf
\end{lstlisting}
\begin{lstlisting}
# vzlist -o smart_name,diskspace.h
 SMARTNAME   DSPACE.H
       101    1113684
       102    1113684
       103    2056788
       104    1113684
\end{lstlisting} 

За подробной информацией о ploop можно обратиться по адресу: \url{https://openvz.org/Ploop/Getting_started}

\clearpage
