\section{Резервное копирование и восстановление}

Утилитами \texttt{vzdump} и \texttt{vzrestore} можно осуществлять резервное копирование и восстановление контейнеров.

Установка \texttt{vzdump} и необходимых зависимостей:
\begin{lstlisting}
# yum install cstream perl-LockFile-Simple
# rpm -ivh "http://ftp.openvz.org/contrib/utils/vzdump/vzdump-1.2-4.noarch.rpm"
Retrieving http://ftp.openvz.org/contrib/utils/vzdump/vzdump-1.2-4.noarch.rpm
Preparing...   ##################################### [100%]
   1:vzdump    ##################################### [100%]
\end{lstlisting}

\subsection{Резервная копия контейнера}
Резервная копия контейнера создается командой:
\begin{lstlisting}
# vzdump 101
\end{lstlisting}
где 101 "--- CTID контейнера.

По умолчанию, копия VPS сохраняется в каталоге \texttt{/vz/dump/}. Размер копии из примера равен 281M:
\begin{lstlisting}
# ls -lh /vz/dump/vzdump-openvz-*.tar 
-rw-r--r-- 1 root root 281M Nov  1 14:02 /vz/dump/vzdump-openvz-101-2014_01_01-12_12_12.tar
\end{lstlisting}

У \texttt{vzdump} есть много параметров, позволяющих осуществлять гибкое резервное копирование\footnote{Подробнее о параметрах можно узнать в \texttt{man vzdump}}.

Создадим резервную копию того же контейнера (CTID = 101), но с использованием дополнительных параметров:
\begin{lstlisting}
# vzdump --suspend --compress --dumpdir /root/ --exclude-path /tmp/ 101
INFO: starting new backup job: vzdump --suspend --compress --dumpdir /root/ --exclude-path /tmp/ 101
INFO: Starting Backup of VM 101 (openvz)
INFO: CTID 101 exist mounted running
INFO: status = CTID 101 exist mounted running
...
INFO: archive file size: 85MB
INFO: Finished Backup of VM 101 (00:00:52)
INFO: Backup job finished successfuly
\end{lstlisting}
где:
\begin{itemize}
    \item \texttt{-{}-suspend} "--- приостановить контейнер;
    \item \texttt{-{}-compress} "--- сжимать архив в формате \texttt{.tgz};
    \item \texttt{-{}-dumpdir} "--- сохранить архив в указанный следующим параметром каталог;
    \item \texttt{-{}-exclude-path}\footnote{В версии \texttt{vzdump-1.2-4} частично не работает параметр \texttt{-{}-exclude-path}} "--- исключить, указанный следующим параметром каталог, из резервной копии.
\end{itemize}

Можно заметить, что в сжатом виде архив весит меньше (86M):
\begin{lstlisting}
# ls -lh /root/vzdump-openvz-101-2014_01_01-12_12_13.tgz
-rw-r--r-- 1 root root 86M Nov  1 14:25 /root/vzdump-openvz-101-2014_01_01-12_12_13.tgz
\end{lstlisting}

Результат операции резервного копирования можно просмотреть в лог-файле, который находится в том же каталоге, что и архив:
\begin{lstlisting}
# tail -4 /root/vzdump-openvz-101-2014_01_01-12_12_13.log
Nov 01 14:25:23 INFO: tar: Exiting with failure status due to previous errors
Nov 01 14:25:23 INFO: archive file size: 85MB
Nov 01 14:25:23 INFO: delete old backup '/root/vzdump-openvz-101-2014_01_01-12_12_13.tgz'
Nov 01 14:25:23 INFO: Finished Backup of VM 101 (00:00:52)
\end{lstlisting}

Чтобы скопировать полученный архив на удаленный сервер по безопасному SSH-соединению, можно воспользоваться утилитой \texttt{scp}:
\begin{lstlisting}
# scp /root/vzdump-openvz-101-2014_01_01-12_12_13.tgz root@backupserver:/backups
root@backupserver's password: b@ckupp@$$wd
vzdump-openvz-101-2014_01_01-12_12_13.tgz  100%  86MB  17.1MB/s  00:05 
\end{lstlisting}

На удаленном сервере можно проверить, скопировался ли архив:
\begin{lstlisting}
root@backupserver:/# ls -lh /backups/
total 86M
-rw-r--r-- 1 root root 86M Nov  1 08:00 vzdump-openvz-101-2014_01_01-12_12_13.tgz
\end{lstlisting}

\subsection{Восстановление контейнера из резервной копии}
Командой \texttt{vzrestore} можно восстановить контейнер с архива. 
\texttt{vzrestore} принимает всего 2 параметра: файл архива и CTID контейнера, в который будет разворачиваться архив.

Создадим контейнер с CTID = 103, содержащий резервную копию контейнера с CTID = 101:
\begin{lstlisting}
# vzrestore /vz/dump/vzdump-openvz-101-2014_01_01-12_12_12.tar 103
INFO: restore openvz backup '/vz/dump/vzdump-openvz-101-2014_01_01-12_12_12.tar' using ID 103
INFO: extracting archive 'vzdump-openvz-101-2014_01_01-12_12_12.tar'
INFO: Total bytes read: 86611520 (86MiB, 26MiB/s)
INFO: extracting configuration to '/etc/vz/conf/103.conf'
INFO: restore openvz backup '/vz/dump/vzdump-openvz-101-2014_01_01-12_12_12.tar' successful
\end{lstlisting}

Проверяем новый контейнер:
\begin{lstlisting}
# vzlist 103
      CTID      NPROC STATUS    IP_ADDR         HOSTNAME
       103          - stopped   192.168.0.101   stud1
\end{lstlisting}

Зададим параметры для нового контейнера:
\begin{lstlisting}
# vzctl set 103 --hostname stud3 --save
CT configuration saved to /etc/vz/conf/103.conf
\end{lstlisting}
\begin{lstlisting}
# vzctl set 103 --ipdel 192.168.0.101 --save
CT configuration saved to /etc/vz/conf/103.conf
\end{lstlisting}
\begin{lstlisting}
# vzctl set 103 --ipadd 192.168.0.103 --save
CT configuration saved to /etc/vz/conf/103.conf
\end{lstlisting}

\begin{lstlisting}
# vzctl start 103
Starting container...
Opening delta /vz/private/103/root.hdd/root.hdd
Adding delta dev=/dev/ploop12539 img=/vz/private/103/root.hdd/root.hdd (rw)
Mounting /dev/ploop12539p1 at /vz/root/103 fstype=ext4 data='balloon_ino=12,' 
Container is mounted
Adding IP address(es): 192.168.0.103
Setting CPU units: 1000
Setting devices
Container start in progress...
\end{lstlisting}

\clearpage
