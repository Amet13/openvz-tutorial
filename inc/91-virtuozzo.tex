\section{OpenVZ 7}

В конце 2014 года компания Odin анонсировала открытие кодовой базы Parallels Cloud Server (проприетарного аналога OpenVZ) и объединение ее с OpenVZ.

В апреле 2015 года был открыт репозиторий с ядром RHEL7 (3.10), в мае были открыты исходные коды пользовательских утилит, а в июне выложены тестовые сборки ISO-образов и RPM-пакеты.

В июле 2016 года анонсирована новая версия OpenVZ 7.

Основные отличия OpenVZ 7 от OpenVZ 6:
\begin{itemize}
    \item OpenVZ 7 базируется на ядре RHEL 7 (3.10);
    \item Замена VEID на UUID, в качестве идентификатора контейнера может использоваться UUID или любое имя;
    \item Управление памятью 4 поколения, использующий memory cgroups;
    \item Поддержка управления виртуальными машинами на базе KVM;
    \item Горячее подключение CPU/RAM для виртуальных машин, поддержка KSM;
    \item Использование \texttt{prlctl} в качестве альтернативы \texttt{vzctl}.
    \item Отказ от развития SimFS в пользу ploop;
    \item Гарантированные лимиты памяти;
    \item Обновленная документация с 2005 года;
    \item Интеграция работы с Docker и OpenStack.
\end{itemize}

Руководство по созданию и управлению контейнерами и виртуальными машинами на базе OpenVZ 7 доступно по адресу: \url{https://github.com/Amet13/vz-tutorial}

\clearpage
