\section{Рекомендации по работе с системой}
\begin{itemize}
    \item Если работа хост-компьютера замедлилась, можно воспользоваться утилитами \texttt{ps}, \texttt{tcpdump}, \texttt{dmesg}, \texttt{*top}\footnote{\texttt{top}, \texttt{htop}, \texttt{atop}, \texttt{iotop}, \texttt{iftop}}, и другими, чтобы узнать, что случилось;
    \item Никогда не перезагружайте компьютер без выяснения обстоятельств неполадок;
    \item Если хост-компьютер был некорректно отключен, при его следующем запуске все разделы должны быть проверены заново и разделы дисков повторно вычислены для каждого VPS;
    \item Не запускайте блобы\footnote{Блоб (англ. binary linked object "--- объект двоичной компоновки) "--- объектный файл без публично доступных исходных кодов, загружаемый в ядро операционной системы} или скрипты, которые принадлежат VPS, непосредственно с хост-компьютера;
    \item Вы должны быть способны обнаружить любой руткит\footnote{Руткит (англ. rootkit) "--- набор программных средств (например, исполняемых файлов, скриптов, конфигурационных файлов), для обеспечения маскировки объектов, контроля и сбора данных} в контейнере. Для этого рекомендуют использовать пакет \texttt{chkrootkit};
    \item Делайте резервные копии (backup) важных данных;
    \item Проверяйте каталог \texttt{/var/log/}, который содержит логи системы;
    \item Используйте \texttt{IPTables}, \texttt{SSH}, \texttt{SSL}, как на хост-компьютере, так и в контейнерах;
    \item Подбирайте сложные для перебора пароли;
    \item Следите за рассылками новостей по безопасности;
    \item Обновляйте ПО, систему и ее компоненты;
    \item Следите за правами пользователей;
    \item Используйте системы мониторинга ресурсов (например Cacti, Munin, MRTG, Zabbix, Nagios);
    \item В случае обнаружения проблем, можно обратиться к документации проектов OpenVZ и Parallels Cloud Server.
\end{itemize}

\clearpage
